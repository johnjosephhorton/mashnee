\documentclass[
10pt, % Main document font size
a4paper, % Paper type, use 'letterpaper' for US Letter paper
oneside, % One page layout (no page indentation)
%twoside, % Two page layout (page indentation for binding and different headers)
headinclude,footinclude, % Extra spacing for the header and footer
BCOR5mm, % Binding correction
]{scrartcl}

\usepackage[
nochapters, % Turn off chapters since this is an article        
beramono, % Use the Bera Mono font for monospaced text (\texttt)
eulermath,% Use the Euler font for mathematics
pdfspacing, % Makes use of pdftex’ letter spacing capabilities via the microtype package
dottedtoc % Dotted lines leading to the page numbers in the table of contents
]{classicthesis} % The layout is based on the Classic Thesis style
\usepackage{arsclassica} % Modifies the Classic Thesis package
\usepackage[T1]{fontenc} % Use 8-bit encoding that has 256 glyphs
\usepackage[utf8]{inputenc} % Required for including letters with accents
\usepackage{graphicx} % Required for including images
\graphicspath{{Figures/}} % Set the default folder for images
\usepackage{enumitem} % Required for manipulating the whitespace between and within lists
\usepackage{lipsum} % Used for inserting dummy 'Lorem ipsum' text into the template
\usepackage{subfig} % Required for creating figures with multiple parts (subfigures)
\usepackage{varioref} % More descriptive referencing

\usepackage{amsmath}

\input{parameters_name.tex}

%----------------------------------------------------------------------------------------
%	HYPERLINKS
%---------------------------------------------------------------------------------------

\hypersetup{
%draft, % Uncomment to remove all links (useful for printing in black and white)
colorlinks=true, breaklinks=true, bookmarks=true,bookmarksnumbered,
urlcolor=webbrown, linkcolor=RoyalBlue, citecolor=webgreen, % Link colors
pdftitle={}, % PDF title
pdfauthor={\textcopyright}, % PDF Author
pdfsubject={}, % PDF Subject
pdfkeywords={}, % PDF Keywords
pdfcreator={pdfLaTeX}, % PDF Creator
pdfproducer={LaTeX with hyperref and ClassicThesis} % PDF producer
}
\hyphenation{Fortran hy-phen-ation} % Specify custom hyphenation points in words with dashes where you would like hyphenation to occur, or alternatively, don't put any dashes in a word to stop hyphenation altogether

%----------------------------------------------------------------------------------------
%	TITLE AND AUTHOR(S)
%----------------------------------------------------------------------------------------

%\title{\normalfont\spacedallcaps{{Comparables to}} } % The article title
\title{\PropertyName{}}
\subtitle{A Comparables analysis} % Uncomment to display a subtitle

\author{\spacedlowsmallcaps{By} \\ \href{https://www.galtongauss.com}{GaltonGauss.com}}
% The article author(s) - author affiliations need to be specified in the AUTHOR AFFILIATIONS block

\date{} % An optional date to appear under the author(s)

%----------------------------------------------------------------------------------------

\begin{document}

%----------------------------------------------------------------------------------------
%	HEADERS
%----------------------------------------------------------------------------------------

\renewcommand{\sectionmark}[1]{\markright{\spacedlowsmallcaps{#1}}} % The header for all pages (oneside) or for even pages (twoside)
%\renewcommand{\subsectionmark}[1]{\markright{\thesubsection~#1}} % Uncomment when using the twoside option - this modifies the header on odd pages
\lehead{\mbox{\llap{\small\thepage\kern1em\color{halfgray} \vline}\color{halfgray}\hspace{0.5em}\rightmark\hfil}} % The header style

\pagestyle{scrheadings} % Enable the headers specified in this block

%----------------------------------------------------------------------------------------
%	TABLE OF CONTENTS & LISTS OF FIGURES AND TABLES
%----------------------------------------------------------------------------------------

\maketitle % Print the title/author/date block

\setcounter{tocdepth}{2} % Set the depth of the table of contents to show sections and subsections only

%% \tableofcontents % Print the table of contents
%% \listoffigures % Print the list of figures
%% \listoftables % Print the list of tables

%----------------------------------------------------------------------------------------
%	ABSTRACT
%----------------------------------------------------------------------------------------

%\section*{Abstract} % This section will not appear in the table of contents due to the star (\section*)
%\lipsum[1] % Dummy text
%----------------------------------------------------------------------------------------
%	AUTHOR AFFILIATIONS
%----------------------------------------------------------------------------------------
%----------------------------------------------------------------------------------------
% \newpage % Start the article content on the second page, remove this if you have a longer abstract that goes onto the second page


%----------------------------------------------------------------------------------------
%	INTRODUCTION
%----------------------------------------------------------------------------------------

\section{Introduction}
This report compares the attributes of \PropertyName{} to a selected group of properties.
The report is based on \NumberOfComps{} comparable properties.
The full raw data is reported in Appendix~\ref{sec:raw_data}. 

\section{Description of the comparable properties}

Figure~\ref{fig:price} shows the recorded price for each of the comparables.
The minimum price in the comparables data is \$\MinPrice{}, while the maximum price is \$\MaxPrice{}.
The \PropertyName{} price [is|is not] within the range. 

\begin{figure}[tb]
\centering
\caption{Reported prices for properties comparable to \PropertyName{}} \label{fig:price}  
\includegraphics[width=0.95\columnwidth]{plots/price.pdf} 
\end{figure}

\begin{figure}[tb]
\centering
\caption{Reported livable square feet for properties comparable to \PropertyName{}} \label{fig:square_feet}  
\includegraphics[width=0.95\columnwidth]{plots/square_feet.pdf} 
\end{figure}

\begin{figure}[tb]
\centering
\caption{Reported bedrooms and bathrooms for properties comparable to \PropertyName{}} \label{fig:bedroom_bathroom}  
\includegraphics[width=0.95\columnwidth]{plots/bedroom_bathroom.pdf} 
\end{figure}

\begin{figure}[tb]
\centering
\caption{Reported bedrooms and bathrooms for properties comparable to \PropertyName{}} \label{fig:bedroom_bathroom}  
\includegraphics[width=0.95\columnwidth]{plots/baths_versus_bedrooms.pdf} 
\end{figure}

\begin{figure}[tb]
\centering
\caption{Overall characteristics of properties comparable to \PropertyName{}} \label{fig:ecdf}  
\includegraphics[width=0.95\columnwidth]{plots/ecdf.pdf} 
\end{figure}


\section{Price versus square footage}

Figure~\ref{fig:price_versus_square_feet} plots the recorded price versus square footage.

\begin{figure}[tb]
\centering
\caption{Price versus square footage} \label{fig:price_versus_square_feet}  
\includegraphics[width=0.95\columnwidth]{plots/price_versus_square_feet.pdf} 
\end{figure}

In this data, the average price per square foot---not including \PropertyName{}---is \$\MeanPricePerFoot{}/$ft^2$.
At the listed price for \PropertyName{}, the price per square foot is \$\MeanPricePerFootFocal{}/$ft^2$.
Among the comparables, each additional square foot is worth another \$\MarginalPricePerFoot{}/$ft^2$. 
This is essentially a measure of how ``steep'' the best-fit line is in Figure~\ref{fig:price_versus_square_feet}.

The price formula based only on square footage would be
\begin{equation}
  \textsc{Price} = \$ \Intercept{} + (\$ \MarginalPricePerFoot{}/\textsc{SqFt}) \times \textsc{SqFt}. \nonumber
\end{equation}
Using this formula, the predicted price for \PropertyName{}, given that is is \PropertySqFt{} \textsc{SqFT}, would be
\begin{align}
\textsc{Predicted Price} & = \$ \Intercept{} + (\$ \MarginalPricePerFoot{}/\textsc{SqFt}) \times \PropertySqFt{}. \\ \nonumber
                         & = \$ \PropertyPredict{} \nonumber 
\end{align}
which is \PctDiff{}\% \ComparePredictedToActual{} than the asking price of \$\PropertyPrice{}.


\renewcommand{\refname}{\spacedlowsmallcaps{References}} % For modifying the bibliography heading
\bibliographystyle{unsrt}
\bibliography{sample.bib} % The file containing the bibliography

\appendix
\section{Raw data} \label{sec:raw_data} 

\end{document}
