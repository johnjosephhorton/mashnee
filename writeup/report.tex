\documentclass[
10pt, % Main document font size
a4paper, % Paper type, use 'letterpaper' for US Letter paper
oneside, % One page layout (no page indentation)
%twoside, % Two page layout (page indentation for binding and different headers)
headinclude,footinclude, % Extra spacing for the header and footer
BCOR5mm, % Binding correction
]{scrartcl}

\usepackage{longtable}


\usepackage[
nochapters, % Turn off chapters since this is an article        
beramono, % Use the Bera Mono font for monospaced text (\texttt)
eulermath,% Use the Euler font for mathematics
pdfspacing, % Makes use of pdftex’ letter spacing capabilities via the microtype package
dottedtoc % Dotted lines leading to the page numbers in the table of contents
]{classicthesis} % The layout is based on the Classic Thesis style
\usepackage{arsclassica} % Modifies the Classic Thesis package
\usepackage[T1]{fontenc} % Use 8-bit encoding that has 256 glyphs
\usepackage[utf8]{inputenc} % Required for including letters with accents
\usepackage{graphicx} % Required for including images
\graphicspath{{Figures/}} % Set the default folder for images
\usepackage{enumitem} % Required for manipulating the whitespace between and within lists
\usepackage{subfig} % Required for creating figures with multiple parts (subfigures)
\usepackage{varioref} % More descriptive referencing


\usepackage{amsmath}

\input{parameters_name.tex}

%----------------------------------------------------------------------------------------
%	HYPERLINKS
%---------------------------------------------------------------------------------------

\hypersetup{
%draft, % Uncomment to remove all links (useful for printing in black and white)
colorlinks=true, breaklinks=true, bookmarks=true,bookmarksnumbered,
urlcolor=webbrown, linkcolor=RoyalBlue, citecolor=webgreen, % Link colors
pdftitle={}, % PDF title
pdfauthor={\textcopyright}, % PDF Author
pdfsubject={}, % PDF Subject
pdfkeywords={}, % PDF Keywords
pdfcreator={pdfLaTeX}, % PDF Creator
pdfproducer={LaTeX with hyperref and ClassicThesis} % PDF producer
}
\hyphenation{Fortran hy-phen-ation} % Specify custom hyphenation points in words with dashes where you would like hyphenation to occur, or alternatively, don't put any dashes in a word to stop hyphenation altogether

%----------------------------------------------------------------------------------------
%	TITLE AND AUTHOR(S)
%----------------------------------------------------------------------------------------

\newcommand{\GG}{$\mathbb{G}\mathbb{G}$}

%\title{\normalfont\spacedallcaps{{Comparables to}} } % The article title
\title{Comparables Analysis of:\\ \PropertyName{}\footnote{\textcopyright  Galton Gauss, LLC. 2019. Feel free to share this report with others, but please do not post at a publicly available URL.}}
\subtitle{} % Uncomment to display a subtitle

\author{\spacedlowsmallcaps{By} \\ \href{https://www.galtongauss.com}{Galton Gauss} \\ $\mathbb{G}\mathbb{G}$}
% The article author(s) - author affiliations need to be specified in the AUTHOR AFFILIATIONS block

\date{} % An optional date to appear under the author(s)

%----------------------------------------------------------------------------------------

\begin{document}

%----------------------------------------------------------------------------------------
%	HEADERS
%----------------------------------------------------------------------------------------

\renewcommand{\sectionmark}[1]{\markright{\spacedlowsmallcaps{#1}}} % The header for all pages (oneside) or for even pages (twoside)
%\renewcommand{\subsectionmark}[1]{\markright{\thesubsection~#1}} % Uncomment when using the twoside option - this modifies the header on odd pages
\lehead{\mbox{\llap{\small\thepage\kern1em\color{halfgray} \vline}\color{halfgray}\hspace{0.5em}\rightmark\hfil}} % The header style

\pagestyle{scrheadings} % Enable the headers specified in this block

%----------------------------------------------------------------------------------------
%	TABLE OF CONTENTS & LISTS OF FIGURES AND TABLES
%----------------------------------------------------------------------------------------

\maketitle % Print the title/author/date block

\setcounter{tocdepth}{2} % Set the depth of the table of contents to show sections and subsections only

%% \tableofcontents % Print the table of contents
%% \listoffigures % Print the list of figures
%% \listoftables % Print the list of tables

%----------------------------------------------------------------------------------------
%	ABSTRACT
%----------------------------------------------------------------------------------------

%\section*{Abstract} % This section will not appear in the table of contents due to the star (\section*)
%\lipsum[1] % Dummy text
%----------------------------------------------------------------------------------------
%	AUTHOR AFFILIATIONS
%----------------------------------------------------------------------------------------
%----------------------------------------------------------------------------------------
% \newpage % Start the article content on the second page, remove this if you have a longer abstract that goes onto the second page


%----------------------------------------------------------------------------------------
%	INTRODUCTION
%----------------------------------------------------------------------------------------

\section{About this report}
This report was prepared by Galton Gauss LLC, using comparable properties provided by the user.
Although the report was carefully prepared, there could be errors.
Was there a problem with one of the comparables imported?
Question about what a number or figure means?
If you have \emph{any} questions or feedback, please email us at \texttt{info@GaltonGauss.com}.

%% \begin{huge}
%% \begin{align}
%%   \mathbb{G}\mathbb{G} \nonumber
%% \end{align}
%% \end{huge}

\section{How can this report be used?}
\begin{itemize}
\item The report can be used to assess whether buyer or seller is making a reasonable offer, and provide some justification for a price. 
\item The report can be used to discover whether the comps reasonable. For example, the report can make it clear whether the target property lies outside the range for the most important attributes. As buyers and sellers try to arrive at a price, aggreeing on comps can be an important first step. 
\item The report can be used to figure out how much various other attributes have to be ``worth'' to justify a differnt price. For example, if the target house has an outdated kitched but the comparables do not, does the listing price discount reflect this difference? 
\end{itemize} 

\section{Methods}
The basic idea behind \href{https://en.wikipedia.org/wiki/Comparables}{comparables} analysis is to find similar properties, but then ``adjust'' for differences.
What should we adjust for and by how much? 
An extra square foot might be more valuable in one neighborhood than an other. 

This report focuses on what economists call ``vertical'' attributes of a property, or things that most people value in a ``more is better'' fashion.
Examples include things like more square footage, more land, more bathrooms, more bedrooms, a newer roof, a newer furnace and so on.
Because these factors are nearly universally valued, more/newer means a higher price.
These are the factors that should be considered when making ``comparables'' more, well, comparable. 

There are also other factors that economists call ``horizontal'' that have an unclear or even non-existent relationship to price because they just reflect tasts.  
Some people like modern houses; some people like the Cape Cod style.
If these factors are associated with price, it is typically just because they are related to some other factor.

Some of the factors considered: 
\begin{itemize}
\item Conditions of sale
\item Financing
\item Market conditions
\item Location comparability
\item Physical comparability 
\end{itemize}

\section{Description of the comparable properties}
This report compares the attributes of \PropertyName{} to \NumberOfComps{} comparable properties.
The full raw data is reported in Appendix~\ref{sec:raw_data}. 

\input{tables/comps.tex}

\subsection{Price range} 
Figure~\ref{fig:price} shows the recorded price for each of the comparables.
The minimum price in the comparables data is \$\MinPrice{}, while the maximum price is \$\MaxPrice{}.
The \PropertyName{} price [is|is not] within the range, and is at the TK percentile (i.e., TK\% have prices less than \PropertyName{}, while TK\% have a higher price).
It is useful that the target house is within the range.
The two closest properties in terms of price to \PropertyName{} are TK and TK. 

\begin{figure}
\centering
\caption{Reported prices for properties comparable to \PropertyName{}} \label{fig:price}  
\includegraphics[width=0.95\columnwidth]{plots/price.pdf} 
\end{figure}

\subsection{Square footage}
An important determinant of price in any real estate market is the square footage.
\PropertyName{} square footage is TK.
The smallest property in the comparables is TK, while the largest is TK, and so \PropertyName{} [is|is not] in the range. 
The two closest properties in size are TK and TK, which are TK\% larger and TK\% smaller than \PropertyName{}, respectively.
To see all the sizes at once, Figure~\ref{fig:square_feet} shows the reported livable square feet for each property.
The square footage for \PropertyName{} is plotted in pink. 

\begin{figure}
\centering
\caption{Reported livable square feet for properties comparable to \PropertyName{}} \label{fig:square_feet}  
\includegraphics[width=0.95\columnwidth]{plots/square_feet.pdf} 
\end{figure}

\subsection{Predicting the price from the square footage}

Figure~\ref{fig:price_versus_square_feet} plots the recorded price versus square footage.
In this data, the average price per square foot---not including \PropertyName{}---is \$\MeanPricePerFoot{}/$ft^2$.
At the listed price for \PropertyName{}, the price per square foot is \$\MeanPricePerFootFocal{}/$ft^2$.

Note that in Figure~\ref{fig:price_versus_square_feet}, there is a sloped line.
This is a \href{https://en.wikipedia.org/wiki/Linear_regression}{regression} line that can tell us, among among the comparables, how much each additional square foot is ``worth.''
In this data, each additional square foot is worth \$\MarginalPricePerFoot{}/$ft^2$. 

\begin{figure}
\centering
\caption{Price versus square footage} \label{fig:price_versus_square_feet}  
\includegraphics[width=0.95\columnwidth]{plots/price_versus_square_feet.pdf} 
\end{figure}

We can use this regression line to make a prediction about the price. 
The price formula based only on square footage would be
\begin{equation}
  \textsc{Price} = \$ \Intercept{} + (\$ \MarginalPricePerFoot{}/\textsc{SqFt}) \times \textsc{SqFt}. \nonumber
\end{equation}
Using this formula, the predicted price for \PropertyName{}, given that is is \PropertySqFt{} \textsc{SqFT}, would be
\begin{align}
\textsc{Predicted Price} & = \$ \Intercept{} + (\$ \MarginalPricePerFoot{}/\textsc{SqFt}) \times \PropertySqFt{}. \\ \nonumber
                         & = \$ \PropertyPredict{} \nonumber 
\end{align}
which is \PctDiff{}\% \ComparePredictedToActual{} than the asking price of \$\PropertyPrice{}.

\subsection{Dealing with uncertainty} 
In any estimate about the future, there is uncertainty.
There are many sources of uncertainty that could cause the prediction for any particular property to differ from the price the property is ultimately sold at.
For example, there could be local market conditions, a motivated seller, unique amenities not captured by relatively crude metrics like square footage---is the house particularly charming? have an interesting history? great views?---and so on. 

These idiosyncratic factors certainly affected the price of the comparables.
This in turn can affect the prediction for the focal property. 
For example, suppose a particularly beautiful but small house was chosen as a comparable.
This small house ``punches above its weight'' having a higher-than-anticipated price, given the size.
This will tend to cause the regression line to be not quite as steep, in turn implying that size doesn't matter as much.
Similarly, a large but ugly that does poorly on the market will also make our regression line not as steep.

To see how our choice of comparables affects our prediction, we can do something called \href{https://en.wikipedia.org/wiki/Bootstrapping_(statistics)}{``bootstrapping.''}
In essence, we create ``new'' sets of comparables from our existing set.
We take our comparables and then ``draw'' from that pool TK times, \emph{with replacement} meaning that one property could show up more than once or not at all.
We then pretend that this bootrap sample of comparables is real and estimate our regression model, just like before.
With a computer, we can do this many, many times and see the range of predictions we get. 

Figure~\ref{fig:bootstrap_price_predictions} plots the distribution of predictions for \PropertyName{}.
It uses samples from the comparable properties with replacement.
The listing price for \PropertyName{} is plotted as a vertical line. 
We can see that 5\% to 95\% is TK.

\begin{figure}
\centering
\caption{Price predictions for \PropertyName{} versus listing price} \label{fig:bootstrap_price_predictions}  
\includegraphics[width=0.55\columnwidth]{plots/bootstrap_price_predictions.pdf} 
\end{figure}

\subsection{Bedrooms and bathrooms}
The square footage alone is not the only factor that matters.
People care a great deal 
\begin{figure}
\centering
\caption{Reported bedrooms and bathrooms for properties comparable to \PropertyName{}} \label{fig:bedroom_bathroom}  
\includegraphics[width=0.95\columnwidth]{plots/bedroom_bathroom.pdf} 
\end{figure}

\begin{figure}
\centering
\caption{Reported bedrooms and bathrooms for properties comparable to \PropertyName{}} \label{fig:baths_versus_bedrooms}  
\includegraphics[width=0.95\columnwidth]{plots/baths_versus_bedrooms.pdf} 
\end{figure}

\begin{figure}
\centering
\caption{Overall characteristics of properties comparable to \PropertyName{}} \label{fig:ecdf}  
\includegraphics[width=0.95\columnwidth]{plots/ecdf.pdf} 
\end{figure}

\subsection{Bedrooms and bathrooms}

\renewcommand{\refname}{\spacedlowsmallcaps{References}} % For modifying the bibliography heading
\bibliographystyle{unsrt}
\bibliography{sample.bib} % The file containing the bibliography

\appendix
\section{Raw data} \label{sec:raw_data} 

\end{document}
