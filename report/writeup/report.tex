
\documentclass[
12pt, % Main document font size
letterpaper, % Paper type, use 'letterpaper' for US Letter paper
oneside, % One page layout (no page indentation)
headinclude,footinclude, % Extra spacing for the header and footer
BCOR5mm, % Binding correction
]{scrartcl}

\usepackage{longtable}
\usepackage[
nochapters, % Turn off chapters since this is an article        
beramono, % Use the Bera Mono font for monospaced text (\texttt)
eulermath,% Use the Euler font for mathematics
pdfspacing, % Makes use of pdftex’ letter spacing capabilities via the microtype package
dottedtoc % Dotted lines leading to the page numbers in the table of contents
]{classicthesis} % The layout is based on the Classic Thesis style
\usepackage{arsclassica} % Modifies the Classic Thesis package
\usepackage[T1]{fontenc} % Use 8-bit encoding that has 256 glyphs
\usepackage[utf8]{inputenc} % Required for including letters with accents
\usepackage{graphicx} % Required for including images
\graphicspath{{Figures/}} % Set the default folder for images
\usepackage{enumitem} % Required for manipulating the whitespace between and within lists
\usepackage{subfig} % Required for creating figures with multiple parts (subfigures)
\usepackage{varioref} % More descriptive referencing


\usepackage{amsmath}

\input{parameters_name.tex}

\hypersetup{
%draft, % Uncomment to remove all links (useful for printing in black and white)
colorlinks=true, breaklinks=true, bookmarks=true,bookmarksnumbered,
urlcolor=webbrown, linkcolor=RoyalBlue, citecolor=webgreen, % Link colors
pdftitle={}, % PDF title
pdfauthor={\textcopyright}, % PDF Author
pdfsubject={}, % PDF Subject
pdfkeywords={}, % PDF Keywords
pdfcreator={pdfLaTeX}, % PDF Creator
pdfproducer={LaTeX with hyperref and ClassicThesis} % PDF producer
}
\hyphenation{Fortran hy-phen-ation} % Specify custom hyphenation points in words with dashes where you would like hyphenation to occur, or alternatively, don't put any dashes in a word to stop hyphenation altogether


\newcommand{\GG}{$\mathbb{G}\mathbb{G}$}

\title{\PropertyName{}\\ \spacedlowsmallcaps{comparable sales analysis}\footnote{
%    \textcopyright
This report was prepared by \href{https://www.galtongauss.com}{Galton Gauss, LLC}.
All rights reserved. 
Questions about any of the comparables, numbers or figures?
Please email us at \texttt{support@GaltonGauss.com}.
}}

%\subtitle{} % Uncomment to display a subtitle
%\author{\spacedlowsmallcaps{By} \\ \href{https://www.galtongauss.com}{Galton Gauss} \\ $\mathbb{G}\mathbb{G}$}

\date{} % An optional date to appear under the author(s)

%----------------------------------------------------------------------------------------

\begin{document}

%----------------------------------------------------------------------------------------
%	HEADERS
%----------------------------------------------------------------------------------------

\renewcommand{\sectionmark}[1]{\markright{\spacedlowsmallcaps{#1}}} % The header for all pages (oneside) or for even pages (twoside)
%\renewcommand{\subsectionmark}[1]{\markright{\thesubsection~#1}} % Uncomment when using the twoside option - this modifies the header on odd pages
\lehead{\mbox{\llap{\small\thepage\kern1em\color{halfgray} \vline}\color{halfgray}\hspace{0.5em}\rightmark\hfil}} % The header style

\pagestyle{scrheadings} % Enable the headers specified in this block

%----------------------------------------------------------------------------------------
%	TABLE OF CONTENTS & LISTS OF FIGURES AND TABLES
%----------------------------------------------------------------------------------------

\maketitle % Print the title/author/date block

\setcounter{tocdepth}{2} % Set the depth of the table of contents to show sections and subsections only

%% \tableofcontents % Print the table of contents
%% \listoffigures % Print the list of figures
%% \listoftables % Print the list of tables

%----------------------------------------------------------------------------------------
%	ABSTRACT
%----------------------------------------------------------------------------------------

%\section*{Abstract} % This section will not appear in the table of contents due to the star (\section*)
%\lipsum[1] % Dummy text
%----------------------------------------------------------------------------------------
%	AUTHOR AFFILIATIONS
%----------------------------------------------------------------------------------------
%----------------------------------------------------------------------------------------
% \newpage % Start the article content on the second page, remove this if you have a longer abstract that goes onto the second page

%----------------------------------------------------------------------------------------
%	INTRODUCTION
%----------------------------------------------------------------------------------------

%\section{About this report}
%% \begin{huge}
%% \begin{align}
%%   \mathbb{G}\mathbb{G} \nonumber
%% \end{align}
%% \end{huge}

\section{Comparable property sales}
This report uses advanced statistical tools and predictive models to provide a graphical, in-depth analysis of 
\textbf{\PropertyName{}, \PropertyCity{}, \PropertyState{}}, 
a \NumberOfBedrooms{}-bedroom, \NumberOfBaths{}-bath 
\PropertyType{} built in \PropertyYearBuilt{}, and \NumberOfComps{} comparable recent property sales:

\input{tables/comps.tex}
Data from comparable recent sales are used to analyze each of the important attributes of the target property, both individually and collectively, enabling parties to evaluate the price in a highly informed manner and negotiate terms with confidence.

\subsection{How to use this report}
\begin{itemize}
\item The report can be used to assess whether a party is making a reasonable offer, and provide justification for a price. 
\item The report can be used to discover whether the selected comparables are reasonable and actually relevant for purposes of comparison. For example, the report can clarify whether a target property lies outside of the range of a given property attribute among the selected comparables. As parties work to arrive at a price, agreeing on comps can be a critical first step. 
\item The report can be used to determine the value of various attributes necessary to justify a price or offer. For example, if a target house has an outdated kitchen but the comparables do not, does the listing price discount reflect this difference? 
\end{itemize} 

For more details on the data and analytical methods used, see Appendix~\ref{sec:methods}.

\subsection{Sale prices} 
Figure~\ref{fig:price} below shows the listing price for \textbf{\PropertyName{}} and the sale price of each comparable property.
The listing price for \PropertyName{} is plotted in pink. 
Among the comparables, the minimum sale price is \$\MinPrice{} and the maximum price is \$\MaxPrice{}.
The listing price for \PropertyName{} \InPriceRange{} within the range of the comparable sales prices and is at the \PricePercentile{}th percentile of those prices.


\begin{figure}[h!]
\centering
\caption{Current listing price for \PropertyName{} and reported sale prices of comparable properties} 
\label{fig:price}  
\includegraphics[width=0.95\columnwidth]{plots/price.pdf} 
\end{figure}


It is useful that the target house is within the comparable sales price range, though it is important to keep in mind that the listing price is not a sale price, and the ultimate sale price may be far away from this initial ask.
In terms of price, the property closest to \PropertyName{} is \ClosestOnPrice{}. 

\subsection{Square footage}
An important determinant of price in any real estate market is square footage.
Among the comparables, the smallest property is \MinSize{}$ft^2$ and the largest is \MaxSize{}$ft^2$---at \PropertySize{}$ft^2$, \PropertyName{} \InSizeRange{} in this range. 
Figure~\ref{fig:square_feet} below shows the square footage of \PropertyName{} and each comparable property.
The square footage of \PropertyName{} is plotted in pink. 
In terms of price, the property closest to \PropertyName{} is \ClosestOnSize{}.

\begin{figure}[h!]
\centering
\caption{Square footage of \PropertyName{} and comparable properties} \label{fig:square_feet}  
\includegraphics[width=0.95\columnwidth]{plots/square_feet.pdf} 
\end{figure}

Figure~\ref{fig:price_versus_square_feet} plots the price versus the square footage of \PropertyName{} and comparable properties.
In this data, the average price per square foot among the comparables---\textit{i.e.,} excluding \PropertyName{}---is \$\MeanPricePerFoot{}/$ft^2$.
At its listed price, the price per square foot for \PropertyName{} is \$\MeanPricePerFootFocal{}/$ft^2$.
%---\MeanPricePerFootPct{}\% \CompareMeanPricePerFoot{} than average of the comparables.

The sloped dashed line in Figure~\ref{fig:price_versus_square_feet} is a \href{https://en.wikipedia.org/wiki/Linear_regression}{regression} line that can tell us how much each additional square foot is ``worth'' in terms of sale price among the comparables.
Among the properties comparable to \PropertyName{}, each additional square foot is worth \$\MarginalPricePerFoot{}/$ft^2$. 

\begin{figure}
\centering
\caption{Price versus square footage} \label{fig:price_versus_square_feet}  
\includegraphics[width=0.95\columnwidth]{plots/price_versus_square_feet.pdf} 
\end{figure}

A positive relationship between square footage and price---\textit{i.e.,} a price that increases with greater square footage---is a good sign that the comparables are reasonable. If additional square footage were associated with a lower price, we might question whether the selected properties were good comparables for \PropertyName{}. 

\subsection{Bedrooms and bathrooms}
Beyond square footage, buyers care about the numbers of bedrooms and bathrooms, though because these quantities are changeable with renovations, overall space is widely regarded as more important.
However, if a buyer is not interested in performing renovations to a property, the number of existing rooms could be quite important.
Figure~\ref{fig:bedroom_bathroom} below shows the numbers of bedrooms and bathrooms of \PropertyName{} and comparable properties. 

\begin{figure}
\centering
\caption{Numbers of bedrooms and bathrooms of properties comparable to \PropertyName{}} \label{fig:bedroom_bathroom}  
\includegraphics[width=0.95\columnwidth]{plots/bedroom_bathroom.pdf} 
\end{figure}

On average, the comparable properties have \AverageBedrooms{} bedrooms and \AverageBaths{} bathrooms. \PropertyName{} has [] bedrooms (\NumberOfBedrooms{}) and [] bathrooms (\NumberOfBaths{}).

\subsection{Age}

\begin{figure}
\centering
\caption{Year built of properties comparable to \PropertyName{}} \label{fig:age}  
\includegraphics[width=0.95\columnwidth]{plots/age.pdf} 
\end{figure}
\subsection{Age}

\subsection{Lot size}

\begin{figure}
\centering
\caption{Lot size of properties comparable to \PropertyName{}} \label{fig:lot_size}  
\includegraphics[width=0.95\columnwidth]{plots/lot_size.pdf} 
\end{figure}


\subsection{Combined attributes}
Having examined certain attributes individually, we next consider the combined attributes of \PropertyName{}.
Figure~\ref{fig:ecdf} below shows how the attributes of \PropertyName{} align with the attributes of each comparable property.
For each numerical value, we show the percentage difference between the attribute of \PropertyName{} and the same attribute of each comparable.

\begin{figure}
\centering
\caption{Combined attributes of properties comparable to \PropertyName{}} \label{fig:ecdf}  
\includegraphics[width=0.95\columnwidth]{plots/ecdf.pdf} 
\end{figure}

We order the properties by the average percentile difference---the a
At a glance, we can see how far away each property is from \PropertyName{} on a number of dimensions. 

\section{Predicted sale price}
Based on the comparables, we can make predictions about the likely sale price for \PropertyName{}. 
To being, we simply use square footage.
After that, we allow for more factors to affect the sale price.

\subsection{From square footage}

The regression line from Figure~\ref{fig:price_versus_square_feet} can be used to make a prediction about the sale price for \PropertyName{}. 
The formula based only on square footage would be
\begin{equation}
  \textsc{Price} = \$ \Intercept{} + (\$ \MarginalPricePerFoot{}/\textsc{SqFt}) \times \textsc{SqFt}. \nonumber
\end{equation}
Using this formula, the predicted price for \PropertyName{}, given that is is \PropertySqFt{} \textsc{SqFT}, would be
\begin{align}
\textsc{Predicted Price} & = \$ \Intercept{} + (\$ \MarginalPricePerFoot{}/\textsc{SqFt}) \times \PropertySqFt{}. \\ \nonumber
                         & = \$ \PropertyPredict{} \nonumber 
\end{align}
which is \PctDiff{}\% \ComparePredictedToActual{} than the asking price of \$\PropertyPrice{}.

With any prediciton, there is uncertainty in the prediciton.
We can account for some of it using a technique called bootstrapping---see Appendix~\ref{sec:bootstrapping} for details. 
Using this method, we get a whole range or price predictions for \PropertyName{}, which we plot as a historgram in Figure~\ref{fig:bootstrap_price_predictions}.
The listing price for \PropertyName{} is plotted as a vertical line. 
We can see that 5\% to 95\% is TK.

\begin{figure}
\centering
\caption{Price predictions for \PropertyName{} versus listing price} \label{fig:bootstrap_price_predictions}  
\includegraphics[width=0.55\columnwidth]{plots/bootstrap_price_predictions.pdf} 
\end{figure}

\subsection{Using all property attributes}
We also fit a model that allows for bedrooms and bathrooms to affect the price, as well as allow the return to more square footage depend on the number of other rooms.
This kind of more flexible model can lead to better predictions. albeit at the cost of more model complexity and greater danger of what is called ``over-fitting'' which is the statistical equilvalent of a hasty generalization.
See Appendix~\ref{sec:regularization} for details. 

After a large search of possible models, the best model to predict the sale price of \PropertyName{} from the comparables is: 
\begin{align}
\mbox{Predicted Price} = \input{formula.tex}.
\end{align}
If we added a new comparable property, we could use this formula to predict their sale price.
This is turn could help us see whehther it is a reasonable comparable---if we find the predicted sale price is wildly different from the actual sale price---either the model is wrong or the property is not actually comparable (say it is a location that radically increases or decreases the price). 

Using our model, we predict the price of all the properties, including \PropertyName{}.
We do this in Figure~\ref{fig:predictive_model}.
The 

\begin{figure}
\centering
\caption{Predictive model predicttions for properties versus prices for \PropertyName{}} \label{fig:predictive_model}  
\includegraphics[width=0.95\columnwidth]{plots/predictive_model.pdf} 
\end{figure}

\renewcommand{\refname}{\spacedlowsmallcaps{References}} % For modifying the bibliography heading
\bibliographystyle{unsrt}
\bibliography{sample.bib} % The file containing the bibliography

\pagebreak

\appendix

\section{Appendix} 

\subsection{Motivation for comparables analysis} \label{sec:methods}
The basic idea behind \href{https://en.wikipedia.org/wiki/Comparables}{comparables} analysis is to find similar properties, but then ``adjust'' for differences, such as square footage.
This raises the question of what what should we adjust for and by how much.
An extra square foot might be more valuable in one neighborhood than an other. 
Perhaps just as important is assessing whether the comparables are actually reasonable for the target property. 

This report focuses on comparing what economists call ``vertical'' attributes of a property, or things that most people value in a ``more is better'' fashion.
Examples include things like more square footage, more land, more bathrooms, more bedrooms, a newer roof, a newer furnace and so on.
Because these factors are nearly universally valued, more/newer means a higher price.
These are the factors that should be considered when making ``comparables'' more, well, comparable. 

There are also other factors that economists call ``horizontal'' that have an unclear or even non-existent relationship to price because they just reflect tasts.  
Some people like modern houses; some people like the Cape Cod style.
If these factors are associated with price, it is typically just because they are related to some other factor.

Some of the factors considered: 
\begin{itemize}
\item Conditions of sale
\item Financing
\item Market conditions
\item Location comparability
\item Physical comparability 
\end{itemize}

\subsection{Dealing with prediction uncertainty}  \label{sec:bootstrapping} 
In any estimate about the future, there is uncertainty.
There are many sources of uncertainty that could cause the prediction for any particular property to differ from the price the property is ultimately sold at.
For example, there could be local market conditions, a motivated seller, unique amenities not captured by relatively crude metrics like square footage---is the house particularly charming? have an interesting history? great views?---and so on. 

These idiosyncratic factors certainly affected the price of the comparables.
This in turn can affect the prediction for the focal property. 
For example, suppose a particularly beautiful but small house was chosen as a comparable.
This small house ``punches above its weight'' having a higher-than-anticipated price, given the size.
This will tend to cause the regression line to be not quite as steep, in turn implying that size doesn't matter as much.
Similarly, a large but ugly that does poorly on the market will also make our regression line not as steep.

To see how our choice of comparables affects our prediction, we can do something called \href{https://en.wikipedia.org/wiki/Bootstrapping_(statistics)}{``bootstrapping.''}
In essence, we create ``new'' sets of comparables from our existing set.
We take our comparables and then ``draw'' from that pool TK times, \emph{with replacement} meaning that one property could show up more than once or not at all.
We then pretend that this bootrap sample of comparables is real and estimate our regression model, just like before.
With a computer, we can do this many, many times and see the range of predictions we get. 

\subsection{Avoiding over-fitting} \label{sec:regularization}
Suppose you meet someone named ``Dave'' who happens to be very rich.
Would you start assuming that everyone you meet named Dave is rich?
Of course not---you would rightly view this as an over-generalization.
You would also rely on your knowledge that there probably is not much of a relationship between name and wealth---unlike say occupation and wealth.
If you learned that Dave was a banker and rich, you might use your knowledge of the world to generalize about future bankers.

Statistical models don't ``know'' what features are general and which are not.
As such, they are prone to over-generalization that is analgous to the rich-Dave example. 
This is called \href{https://en.wikipedia.org/wiki/Overfitting}{``over-fitting.''}
To prevent this, we can do a number of things, including penalizing models that are too complicted.
If we use both name and occupation to make a prediction about Dave, we'd find that getting rid of the name part and keeping the occupation part is helpful, especially since the occupation part is the better predictor. 

Among our comparables, we can leave one comparable out and try to predict the price.
As we know the price for this, we can assess how well our model does.
Doing this many, many times, we can learn what's the best model we have, subject to the data we have.
This is called \href{https://en.wikipedia.org/wiki/Cross-validation_(statistics)}{``cross-validation.''}

%\section{Raw data} \label{sec:raw_data} 

\end{document}
