\documentclass[
12pt, % Main document font size
letterpaper, % Paper type, use 'letterpaper' for US Letter paper
oneside, % One page layout (no page indentation)
headinclude,footinclude, % Extra spacing for the header and footer
BCOR5mm, % Binding correction
]{scrartcl}

\usepackage{longtable}
\usepackage[
nochapters, % Turn off chapters since this is an article        
beramono, % Use the Bera Mono font for monospaced text (\texttt)
eulermath,% Use the Euler font for mathematics
pdfspacing, % Makes use of pdftex’ letter spacing capabilities via the microtype package
dottedtoc % Dotted lines leading to the page numbers in the table of contents
]{classicthesis} % The layout is based on the Classic Thesis style
\usepackage{arsclassica} % Modifies the Classic Thesis package
\usepackage[T1]{fontenc} % Use 8-bit encoding that has 256 glyphs
\usepackage[utf8]{inputenc} % Required for including letters with accents
\usepackage{graphicx} % Required for including images
\graphicspath{{Figures/}} % Set the default folder for images
\usepackage{enumitem} % Required for manipulating the whitespace between and within lists
\usepackage{subfig} % Required for creating figures with multiple parts (subfigures)
\usepackage{varioref} % More descriptive referencing


\usepackage{amsmath}

\input{parameters_name.tex}

\hypersetup{
%draft, % Uncomment to remove all links (useful for printing in black and white)
colorlinks=true, breaklinks=true, bookmarks=true,bookmarksnumbered,
urlcolor=webbrown, linkcolor=RoyalBlue, citecolor=webgreen, % Link colors
pdftitle={}, % PDF title
pdfauthor={\textcopyright}, % PDF Author
pdfsubject={}, % PDF Subject
pdfkeywords={}, % PDF Keywords
pdfcreator={pdfLaTeX}, % PDF Creator
pdfproducer={LaTeX with hyperref and ClassicThesis} % PDF producer
}
\hyphenation{Fortran hy-phen-ation} % Specify custom hyphenation points in words with dashes where you would like hyphenation to occur, or alternatively, don't put any dashes in a word to stop hyphenation altogether


\newcommand{\GG}{$\mathbb{G}\mathbb{G}$}

\title{\PropertyName{}\\ \spacedlowsmallcaps{comparable market analysis}\footnote{
%    \textcopyright
This report was prepared by \href{https://www.galtongauss.com}{Galton Gauss, LLC}.
All rights reserved. 
Questions about any of the comparables, numbers or figures?
Please email us at \texttt{support@GaltonGauss.com}.
$\mathbb{G}\mathbb{G}$
}}

%\subtitle{} % Uncomment to display a subtitle
%\author{\spacedlowsmallcaps{By} \\ \href{https://www.galtongauss.com}{Galton Gauss} \\ $\mathbb{G}\mathbb{G}$}

\date{} % An optional date to appear under the author(s)

%----------------------------------------------------------------------------------------

\begin{document}

%----------------------------------------------------------------------------------------
%	HEADERS
%----------------------------------------------------------------------------------------

\renewcommand{\sectionmark}[1]{\markright{\spacedlowsmallcaps{#1}}} % The header for all pages (oneside) or for even pages (twoside)
%\renewcommand{\subsectionmark}[1]{\markright{\thesubsection~#1}} % Uncomment when using the twoside option - this modifies the header on odd pages
\lehead{\mbox{\llap{\small\thepage\kern1em\color{halfgray} \vline}\color{halfgray}\hspace{0.5em}\rightmark\hfil}} % The header style

\pagestyle{scrheadings} % Enable the headers specified in this block

%----------------------------------------------------------------------------------------
%	TABLE OF CONTENTS & LISTS OF FIGURES AND TABLES
%----------------------------------------------------------------------------------------

\maketitle % Print the title/author/date block

\setcounter{tocdepth}{2} % Set the depth of the table of contents to show sections and subsections only

%% \tableofcontents % Print the table of contents
%% \listoffigures % Print the list of figures
%% \listoftables % Print the list of tables

%----------------------------------------------------------------------------------------
%	ABSTRACT
%----------------------------------------------------------------------------------------

%\section*{Abstract} % This section will not appear in the table of contents due to the star (\section*)
%\lipsum[1] % Dummy text
%----------------------------------------------------------------------------------------
%	AUTHOR AFFILIATIONS
%----------------------------------------------------------------------------------------
%----------------------------------------------------------------------------------------
% \newpage % Start the article content on the second page, remove this if you have a longer abstract that goes onto the second page

%----------------------------------------------------------------------------------------
%	INTRODUCTION
%----------------------------------------------------------------------------------------

%\section{About this report}
%% \begin{huge}
%% \begin{align}
%%   \mathbb{G}\mathbb{G} \nonumber
%% \end{align}
%% \end{huge}

\section{Comparable property sales}
This report analyzes a number of recently sold properties that are comparable to a target property:
\textbf{\PropertyName{}, \PropertyCity{}, \PropertyState{}}, a \NumberOfBedrooms{}-bedroom, \NumberOfBaths{}-bath  \PropertyType{} built in \PropertyYearBuilt{}.
\TypeWarning{}
The analysis is based on \NumberOfComps{} comparable properties described in the table below:

\begin{small}
\input{tables/comps.tex}
\end{small}

The table reports the street address (which we will use a short-hand when referring to a property), the year it was built, the lot and living area size (both in square feet), the reported number of bedrooms and bathrooms, and finally the sale price (for comparables) or the listed price, for \PropertyName{}. 

\subsection{What makes for a good comparables analysis?}
The quality of this analysis depends critically on the quality of the comparables.
A focus of the analysis will be showing how comparable---or not---these comparables are to the target property. 
We will use the comparables to make predictions about the price for the tareget property, but these predictions are not substitutes for judgement and contextual detail that a realtor provides.
When the report does make predictions, how, precisely, these predictions are made is explained.
This is in keeping with our philosophy that ``black box'' predictions are typically harmful in the sale process. 

\subsection{How to use this report}
\begin{itemize}
\item The report can be used to assess whether a party is making a reasonable offer, and provide justification for a price. 
\item The report can be used to discover whether the selected comparables are reasonable and relevant for purposes of comparison. 
For example, the report can clarify whether the collective attributes of a target property are sufficiently within the ranges of the selected comparables. As parties work to arrive at a price, agreeing on comps can be a critical first step. 
\item The report can be used to determine the value of various attributes necessary to justify a price or offer. For example, if a target house has an outdated kitchen but the comparables do not, does the listing price discount reflect this difference? 
\end{itemize} 

In keeping with our ``no black boxes'' philosophy, each component of the analysis is explained in detail. 
%Additional information about the data and methods used are provided in Appendix~\ref{sec:methods}.

\subsection{Location}
In any comparables analysis, perhaps the most important factor is finding properties in truly comparable locations.
The location of a property strongly affects schools, property taxes, local amenities, crime, walkability and so on.
As such, there is enormous variation in house prices explained purely by location. 
As such, this analysis hinges about the realtor or whoever selected the comparables to pick properties that are all in more or less the same location.
It is much more feasible for a model to account for differences due to lot sizes or square footage among properties that are broadly similar than to try to account for the differences due to geography. 
The old saying that real estate is about ``location, location, location'' is surely true---but hopefully with good comparables, this factor is off the table.

The location of the target property and the comparables are plotted in Figure~\ref{fig:map}.
The location of the comparable properties is plotted in blue and the target property is in pink. 

\begin{figure}[h!]
\centering
\caption{Locations of comparable properties}  \label{fig:map}  
\includegraphics[width=0.65\columnwidth]{plots/map.pdf} 
\end{figure}

On average, the comparable properties are \AvgDistance{} miles away, as the crow flies.
The closest comparable property is only \MinDistance{} miles away, while the farthest away is \MaxDistance{}. 
\States{}
\Cities{}

\subsection{Sale prices and \PropertyName{} listing price}

Figure~\ref{fig:price} below shows the listing price for \textbf{\PropertyName{}} and the sale price of each comparable property.
The listing price for \PropertyName{} is plotted in pink. 
Among the comparables, the minimum sale price is \$\MinPrice{} and the maximum price is \$\MaxPrice{}.
The listing price for \PropertyName{} \InPriceRange{} within the range of the comparable sales prices. 
%and is at the \PricePercentile{}th percentile of those prices.

\begin{figure}[htb]
\centering
\caption{Current listing price for \PropertyName{} and reported sale prices of comparable properties} 
\label{fig:price}  
\includegraphics[width=0.95\columnwidth]{plots/price.pdf} 
\end{figure}

It is useful that the target house is within the comparable sales price range, though it is important to keep in mind that the listing price is not a sale price, and the ultimate sale price may be far away from this initial listing price.
In terms of price, the property closest to \PropertyName{} is \ClosestOnPrice{}. 

\subsection{Square feet of living space}
An important determinant of price in any real estate market is the amount of livable space.
Of course, not all square footage is the same.
A luxury home with custom finishes, high-quality building materials, hardwood floors, architectual details and so on, is not worth the same as a same-sized home with carpeting, lineoleum bathrooms and hollow-core doors.
Again, these kinds of differences are not captured in a model and it is incumbent upon the realtor to make sure that the comparables truly are comparable to the target property. 

Among the comparables, the smallest property is \MinSize{}$ft^2$ and the largest is \MaxSize{}$ft^2$.
At \PropertySize{}$ft^2$, \PropertyName{} \InSizeRange{} in this range. 
Figure~\ref{fig:square_feet} below shows the square footage of \PropertyName{} and each comparable property.
The square footage of \PropertyName{} is plotted in pink. 
In terms of price, the property closest to \PropertyName{} is \ClosestOnSize{}.

\begin{figure}[h!]
\centering
\caption{Square footage of \PropertyName{} and comparable properties} \label{fig:square_feet}  
\includegraphics[width=0.95\columnwidth]{plots/square_feet.pdf} 
\end{figure}

\subsection{Price per per square foot}

Figure~\ref{fig:price_versus_square_feet} plots the price versus the square footage of \PropertyName{} and comparable properties.
In this data, the average price per square foot among the comparables is \$\MeanPricePerFoot{}/$ft^2$.
At its listed price, the price per square foot for \PropertyName{} is \$\MeanPricePerFootFocal{}/$ft^2$,
which is \MeanPricePerFootPct{}\% \ComparePricePerFoot{} than the average price per square foot among the comparables.
The grey area in the plot indicates a measure of how confident the model is in that area of the curve. 

The sloped dashed line in Figure~\ref{fig:price_versus_square_feet} is a \href{https://en.wikipedia.org/wiki/Linear_regression}{regression} line that can tell us how much each additional square foot is ``worth'' in terms of sale price among the comparables.
Among the properties comparable to \PropertyName{}, each additional square foot is worth \$\MarginalPricePerFoot{}/$ft^2$. 

\begin{figure}
\centering
\caption{Price versus square footage} \label{fig:price_versus_square_feet}  
\includegraphics[width=0.85\columnwidth]{plots/price_versus_square_feet.pdf} 
\end{figure}

A positive relationship between square footage and price---i.e., a price that increases with greater square footage---is a good sign that the comparables are reasonable.
If additional square footage were associated with a lower price, we might question whether the selected properties were good comparables for \PropertyName{}. 

\subsection{Bedrooms and bathrooms}
Beyond square footage, buyers care about the numbers of bedrooms and bathrooms.
However, because these quantities are changeable with renovations, overall space is widely regarded as more important.
Keeping a square footage fixed, adding aother bedroom just more finely divides the space already available, potentially leading to cramped quarters. 
Furthermore, each additional bedroom or bathroom might be worth less than the previous one.

Figure~\ref{fig:bedroom_bathroom} below shows the numbers of bedrooms and bathrooms of \PropertyName{} and comparable properties. 
In the figure, the properties are listed from most to fewest bedrooms.
The number of bedrooms is plotted in the left facet, and the number of bathrooms is plotted in the right facet. 
The comparable properties have, on average, \AverageBedrooms{} bedrooms and \AverageBaths{} bathrooms.
\PropertyName{} has bedrooms \NumberOfBedrooms{} and \NumberOfBaths{} bathrooms.

\begin{figure}[!]
\centering
\caption{Numbers of bedrooms and bathrooms of properties comparable to \PropertyName{}} \label{fig:bedroom_bathroom}  
\includegraphics[width=0.95\columnwidth]{plots/bedroom_bathroom.pdf} 
\end{figure}

\subsection{Lot size}
In addition to living space, another determinant of the price is the lot size.
All else equal, people generally prefer more land to less.
However, it matters a great deal what the status of that land is---some properties have land that could be sub-divided, creating another (valuable) house lot;
other parcels of land may not be buildable, though the contribute to the owner's enjoyment of the land, through greater quiet and privacy.

\begin{figure}[!]
\centering
\caption{Lot size (in acres) of properties comparable to \PropertyName{}} \label{fig:lot_size}  
\includegraphics[width=0.95\columnwidth]{plots/lot_size.pdf} 
\end{figure}

The enjoyment of land in turn depends on factors not captured in a model or just the acreage for that matter, such as the shape of the parcel, setting, views, what it abutts and so on.
For example, owners typically value land behind their house more than land in front of their house; 
Land subject to vernal pools or easements of various kinds will be less valued;
Land that is mostly frontage on a busy street will be less valuable, and so on.

These caveats aside, in Figure~\ref{fig:lot_size} plots the lot size for each of the properties. 
Properties are listed from largest to smallest.
The smallest lot is \SmallestLot{} acres, while the largest lot is \LargestLot{} acres.
The target property has a lot size of \PropertyLotSize{} acres.
To help with intrepreation, note that a US football field is about 1.3 acres.
\ExtremeWarningLargestLot{}
\ExtremeWarningSmallestLot{}

\subsection{Year of construction}
The age of construction can be misleading---there are centuries-old houses in excellent condition and twenty year old houses that were poorly maintained and are now ``tear-downs.''
Furthermore, the date of the last rennovation is probably more relevant than the year of construction in assessing in how ``dated'' a property might look. 
However, all else equal, people tend to prefer newer construction, in part because of lower maintenance costs, greater energe efficiency, and so on. 
Figure~\ref{fig:age} shows the year when each property was built, with properties ordered from newest to oldest. 

\begin{figure}
\centering
\caption{Year built of properties comparable to \PropertyName{}} \label{fig:age}  
\includegraphics[width=0.95\columnwidth]{plots/age.pdf} 
\end{figure}

The oldest property in the comparable set was built in \Oldest{}, while the newest is \Youngest{}.
The target property was built in \PropertyYearBuilt{}. 
\ExtremeWarningAgeYoung{}
\ExtremeWarningAgeOld{}

\subsection{Assessing comparability with respect to all attributes}
As we have re-iterated many times, the key to a good comparables analysis is having comparables that are truly comparable.
Although we have looked at each attribute individually---square footage, lot size, age, and so on---we now consider all of them collectively in a single plot.

Having examined key attributes individually, we next consider the combined attributes of \PropertyName{}.
In Figure~\ref{fig:ecdf}, each of the comparable properties is listed in its own facet.
For each attribute, we compute the percentage difference between that comparable proeprty and \PropertyName{}.
The x-axis is centered at 0.  
If the comparable is larger on the dimension, the bar is positive and shaded green;
if the comparare is smaller, then bar is negative and shared red. 

\begin{figure}
\centering
\caption{Combined attributes of properties comparable to \PropertyName{}} \label{fig:ecdf}  
\includegraphics[width=0.95\columnwidth]{plots/ecdf.pdf} 
\end{figure}

Comparable propeties are ordered by the average value of the percentage difference, from high to low. 
With good set of comparables, ideally, the percentage differences would all be fairly small.
If there are large differences, ideally there would be a mix of green and red, both ``verically'' and ``horizontally.'' 

%% \begin{figure}
%% \centering
%% \caption{Combined attributes of properties comparable to \PropertyName{}} \label{fig:hist}  
%% \includegraphics[width=0.95\columnwidth]{plots/hist.pdf} 
%% \end{figure}

\section{Predicted sale price}
The basic idea behind \href{https://en.wikipedia.org/wiki/Comparables}{comparables} analysis is to find similar properties, but then ``adjust'' for differences, such as square footage to make a prediction.
This raises the question of what what should we adjust for and by how much.
An extra square foot might be more valuable in one neighborhood than an other. 
Perhaps just as important is assessing whether the comparables are actually reasonable for the target property. 

This report focuses on comparing what are sometimes called ``vertical'' attributes of a property, or things that most people value in a ``more is better'' fashion.
Examples include things like more square footage, more land, more bathrooms, more bedrooms, a newer roof, a newer furnace and so on.
Because these factors are nearly universally valued, more/newer means a higher price.
These are the factors that should be considered when making ``comparables'' more, well, comparable. 

There are also other factors that economists call ``horizontal'' that have an unclear or even non-existent relationship to price because they just reflect tasts.  
Some people like modern houses; some people like the Cape Cod style.
If these factors are associated with price, it is typically just because they are related to some other factor.

Based on the comparables, we can make predictions about the likely sale price for \PropertyName{}. 
To begin, we use a very simple model, in that we adjust for square footage.
After that, we allow for more factors to affect the price.

Our basic approach is essentially a simple version of what are called Automated Valuation Models (AVMs).
Unlike our model, however, most of these models are inherently a ``black box'' with the details of how predictions are being being either proprietary or too complex to explain.
In truth, the most complex models may not really be understandable even by their creators. 

With these kinds of models, consumers of the model are taking the predictions on the company's say-so.\footnote{\href{https://www.washingtonpost.com/news/where-we-live/wp/2014/06/10/how-accurate-is-zillows-zestimate-not-very-says-one-washington-area-agent/?noredirect=on&utm_term=.e3453c7b3677}{How accurate is Zillow's Zestimate? Not very, says one Washington-area agent.}
}
Our view is that black box models have a real downside relative to simpler, transparent models.
With simple models, it is possible to show how preditions are derives.
This in turn enables parties to more confidently arrive at a price.

With a simple model, buyers, sellers and realtors can discuss the comparables and the predictions.
If the model is too low because it doesn't reflect the value of a new kitchen, it is better for all parties to understand why the model isn't working (kitchen renovation is not used as a predictor), rather wonder how that information may or may not be factored in.

In our simple models, we also purposefully exclude listing prices in our predictions, as these are choices made by parities.
If models systematically used this information to generated predictions, parties would have an incentive to manipulate their listing price to try to influence the bargain. 

While more complex models might be slightly more accurate, as they are based on more factors, these also rapidly run into diminishing returns with more data.
After accounting for location, square footage and lot size, there are not lots of widely available factors that will matter a great deal, particularly in a set of already carefully choosing comparables.

\subsection{A very simple model: predicting price from square feet of living space}

The regression line from Figure~\ref{fig:price_versus_square_feet} can be used to make a prediction about the sale price for \PropertyName{}. 
The formula based only on square footage would be
\begin{equation}
  \textsc{Price} = \$ \Intercept{} + (\$ \MarginalPricePerFoot{}/\textsc{SqFt}) \times \textsc{SqFt}. \nonumber
\end{equation}
Using this formula, the predicted price for \PropertyName{}, given that is is \PropertySqFt{} \textsc{SqFT}, would be
\begin{align}
\textsc{Predicted Price} & = \$ \Intercept{} + (\$ \MarginalPricePerFoot{}/\textsc{SqFt}) \times \PropertySqFt{}. \\ \nonumber
                         & = \$ \PropertyPredict{} \nonumber 
\end{align}
which is \PctDiff{}\% \ComparePredictedToActual{} than the asking price of \$\PropertyPrice{}.

With any prediciton, there is uncertainty in the prediciton.
We can account for some of it using a technique called bootstrapping---see Appendix~\ref{sec:bootstrapping} for details. 
Using this method, we get a whole range or price predictions for \PropertyName{}, which we plot as a historgram in Figure~\ref{fig:bootstrap_price_predictions}.
We exclude estimates below the 5th percentile and above the 95th percentile. 
The listing price for \PropertyName{} is plotted as a vertical line. 
%We can see that 5\% to 95\% is TK.

\begin{figure}
\centering
\caption{Price predictions for \PropertyName{} versus listing price} \label{fig:bootstrap_price_predictions}  
\includegraphics[width=0.55\columnwidth]{plots/bootstrap_price_predictions.pdf} 
\end{figure}

\subsection{Using more property attributes in prediction}
We also fit a model that allows for bedrooms and bathrooms to affect the price, as well as allow the return to more square footage depend on the number of other rooms.
This kind of more flexible model can lead to better predictions. albeit at the cost of more model complexity and greater danger of what is called ``over-fitting'' which is the statistical equilvalent of a hasty generalization.
See Appendix~\ref{sec:regularization} for details. 

After a large search of possible models, the best model to predict the sale price of \PropertyName{} from the comparables is: 
\begin{align}
\mbox{Predicted Price} = \input{formula.tex}.
\end{align}
If we added a new comparable property, we could use this formula to predict their sale price.
This is turn could help us see whehther it is a reasonable comparable---if we find the predicted sale price is wildly different from the actual sale price---either the model is wrong or the property is not actually comparable (say it is a location that radically increases or decreases the price). 

Using our model, we predict the price of all the properties, including \PropertyName{}.
We do this in Figure~\ref{fig:predictive_model}, which will take a moment to explain.
This is called a slope graph.
In the left side, each line starts at the actual sale price (or the listing price in the case \PropertyName{}).
The line then slopes to the right side, which is the predicted sale price for that property.
Each point on the right side is labeled with the percentage change. 

\begin{figure}
\centering
\caption{Predictive model predicttions for properties versus prices for \PropertyName{}} \label{fig:predictive_model}  
\includegraphics[width=0.95\columnwidth]{plots/predictive_model.pdf} 
\end{figure}

There are other automated valuation methods available online, though many simply serve as lead generation services.
There \href{https://www.fhfa.gov/DataTools/Tools/Pages/HPI-Calculator.aspx}{Federal Housing Finance Authority has an online tool} that helps project a price recorded in the past to some date in the future, under the assumption that the property appreciated at the same rate as other properties. 

\renewcommand{\refname}{\spacedlowsmallcaps{References}} % For modifying the bibliography heading
\bibliographystyle{unsrt}
\bibliography{sample.bib} % The file containing the bibliography

\pagebreak

\appendix

\section{Appendix} 

\subsection{Dealing with prediction uncertainty}  \label{sec:bootstrapping} 
In any estimate about the future, there is uncertainty.
There are many sources of uncertainty that could cause the prediction for any particular property to differ from the price the property is ultimately sold at.
For example, there could be local market conditions, a motivated seller, unique amenities not captured by relatively crude metrics like square footage---is the house particularly charming? have an interesting history? great views?---and so on. 

These idiosyncratic factors certainly affected the price of the comparables.
This in turn can affect the prediction for the focal property. 
For example, suppose a particularly beautiful but small house was chosen as a comparable.
This small house ``punches above its weight'' having a higher-than-anticipated price, given the size.
This will tend to cause the regression line to be not quite as steep, in turn implying that size doesn't matter as much.
Similarly, a large but ugly that does poorly on the market will also make our regression line not as steep.

To see how our choice of comparables affects our prediction, we can do something called \href{https://en.wikipedia.org/wiki/Bootstrapping_(statistics)}{``bootstrapping.''}
In essence, we create ``new'' sets of comparables from our existing set.
We take our comparables and then ``draw'' from that pool many times, \emph{with replacement} meaning that one property could show up more than once or not at all.
We then pretend that this bootrap sample of comparables is real and estimate our regression model, just like before.
With a computer, we can do this many, many times and see the range of predictions we get. 

\subsection{Avoiding over-fitting} \label{sec:regularization}
Suppose you meet someone named ``Dave'' who happens to be very rich.
Would you start assuming that everyone you meet named Dave is rich?
Of course not---you would rightly view this as an over-generalization.
You would also rely on your knowledge that there probably is not much of a relationship between name and wealth---unlike say occupation and wealth.
If you learned that Dave was a banker and rich, you might use your knowledge of the world to generalize about future bankers.

Statistical models don't ``know'' what features are general and which are not.
As such, they are prone to over-generalization that is analgous to the rich-Dave example. 
This is called \href{https://en.wikipedia.org/wiki/Overfitting}{``over-fitting.''}
To prevent this, we can do a number of things, including penalizing models that are too complicted.
If we use both name and occupation to make a prediction about Dave, we'd find that getting rid of the name part and keeping the occupation part is helpful, especially since the occupation part is the better predictor. 

Among our comparables, we can leave one comparable out and try to predict the price.
As we know the price for this, we can assess how well our model does.
Doing this many, many times, we can learn what's the best model we have, subject to the data we have.
This is called \href{https://en.wikipedia.org/wiki/Cross-validation_(statistics)}{``cross-validation.''}

%\section{Raw data} \label{sec:raw_data} 

\end{document}
